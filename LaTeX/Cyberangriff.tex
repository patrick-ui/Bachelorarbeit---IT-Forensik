\begin{otherlanguage}{ngerman}
\subsection{Cyberangriff}
Die Bedrohung durch Cyberangriffe nimmt von Jahr zu Jahr zu. Parallel mit der Zunahme ist auch die Häufigkeit der erfassten Fälle gestiegen. Jedoch kann nur nahezu ein Drittel der Vorfälle durch das Bundeskriminalamt (BKA) aufgeklärt werden.\newline
Eine Form des Cyberangriffs sind Advanced Persistent Threads (APTs), bei dem die Form
und die Fokussierung auf ein Ziel, sowie auch die Vorgehensweise besonders
sind[2]. Die Anzahl an APTs steigt von Jahr zu Jahr. Dabei ist auch der Trend zu beob-
achten, dass sie immer mehr an Qualität zunehmen. Hinter APTs stecken oft große Hackergruppen. Dieser Fakt lässt sich daraus ziehen, dass die Komplexität des Angriffes sehr groß ist. Diese größeren Hackergruppen handeln entweder aus wirtschaftlichem oder politischem Interesse. \newline
Zuerst wird das Ziel beobachtet und nach einer Schwachstelle gesucht. Anschließend
probiert der Angreifer in der Zielorganisation einzudringen. Dies geschieht sehr oft
über Phishing, das Ausnutzen von Zero-Day-Exploits oder über direkte Kontakte die
durch soziale Manipulation geknüpft wurden. Wenn der Angreifer einmal im Zielnetz-
werk ist, bewegt er sich quer durch das Netzwerk um sein Zielsystem zu finden. Sobald
er diese findet, geht er in das Sammeln von Informationen über, was vom Säubern der
Tat gefolgt wird. Beim ganzen Prozess wird sehr streng darauf geachtet nicht aufzufal-
len und alle Spuren zu verwischen. So kommt es dazu, dass die Aufklärung solcher
Fälle Monate oder auch ganze Jahre dauern kann. Oft werden die Taten dann ent-
weder durch den Angreifer selbst oder durch externe sogenannte \dq ethical Hacker \dq{} aufgedeckt. In
wenigen Fällen finden die Firmen selber die Schwachstelle.
\end{otherlanguage}
