\begin{otherlanguage}{ngerman}
\subsection{Malware}
\subsubsection{Was ist Malware und wie funktioniert sie?}

Laut Defintion ist Malware eine "bösartige Software"
(auf Englisch "malicious software").
\newline Allgemein dient der Begriff Malware als Klassifizierung von Dateien oder Software, die Schäden verursachen, sobald sie sich im System des Benutzers befinden. 
\newline Die häufigsten Arten sind:
\begin{enumerate}
    \item Viren
    \item Trojaner
    \item Würmer
    \item Spyware
    \item Scareware
    \item Ransomware
    \item Zombie-Malware
    \item Root kit
\end{enumerate}
Dabei können sowohl persönliche Schäden als auch Sachschäden entstehen.

Um einen solchen Malware Angriff zu tätigen, benötigt man einen Schadcode. 
\newline Mithilfe dessen verschaffen sich Cyberkriminelle Zugriff auf das System des Opfers und klauen Passwörter oder andere sensible Daten.
\newline Um den Datendiebstahl durchzuführen, muss ein Cyberkrimineller dafür sorgen, dass der Schadcode auf dem Zielsystem ausgeführt wird. 
\newline Damit das klappt, betten diese den schädlichen Code in eine Datei ein, die das Opfer öffnen soll. 
\newline Nach der Durchführung erfüllt der Schadcode seinen Zweck und installiert zum Beispiel eine Backdoor oder startet einen Key-Logger. Key Logger können unter anderem die Eingabe von Passwörtern protokollieren.

\subsubsection{Arten von Malware}
Neue Schadprogramm Varianten entstehen, wenn im Programmcode Änderungen vorgenommen werden. Wichtig ist hierbei das der Hashwert einzigartig ist, um es auch eine neue Malware nennen zu können. 
\newline Während für bekannte Schadprogramm-Varianten Detektionsmethoden existieren, sind neue Varianten unmittelbar nach ihrem Auftreten unter Umständen noch nicht als Schadprogramme erkennbar und daher besonders bedrohlich. 
\newline 
\begin{enumerate}
    \item Viren
    \item Trojaner
    \item Würmer
    \item Spyware
    \item Scareware
    \item Ransomware
    \item Zombie-Malware
    \item Root kit
\end{enumerate}

\end{otherlanguage}