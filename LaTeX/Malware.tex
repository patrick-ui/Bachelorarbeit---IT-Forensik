\begin{otherlanguage}{ngerman}
\subsection{Malware}
\subsubsection{Was ist Malware und wie funktioniert sie?}

Laut Defintion ist Malware eine "bösartige Software"
(auf Englisch "malicious software").
\newline Allgemein dient der Begriff Malware als Klassifizierung von Dateien oder Software, die Schäden verursachen, sobald sie sich im System des Benutzers befinden. 
\newline Die häufigsten Arten sind:
\begin{enumerate}
    \item Viren
    \item Trojaner
    \item Würmer
    \item Spyware/Adware
    \item Scareware
    \item Ransomware
    \item Root kit und Backdoors
\end{enumerate}
Dabei können sowohl persönliche Schäden als auch Sachschäden entstehen.

Um einen solchen Malware Angriff zu tätigen, benötigt man einen Schadcode. 
Mithilfe dessen verschaffen sich Angreifer Zugriff auf das System des Opfers und klauen Passwörter oder andere sensible Daten.
Um den Datendiebstahl durchzuführen, muss ein Angreifer dafür sorgen, dass der Schadcode auf dem Zielsystem ausgeführt wird. 
Damit das klappt, betten diese den schädlichen Code in eine Datei ein, die das Opfer öffnen soll.
Nach der Durchführung erfüllt der Schadcode seinen Zweck und installiert zum Beispiel eine Backdoor oder startet einen Key-Logger. Key Logger können unter anderem die Eingabe von Passwörtern protokollieren.

\subsubsection{Arten von Malware}
Neue Schadprogramm Varianten entstehen, wenn die Funktionsweise der neuen Malware grundlegend von bereits vorhandenen Schadprogrammen abweicht. Wichtig ist hierbei, dass der Hashwert einzigartig ist, aber ein neuer Hashwert nicht mit einer neuen Malware gleichzusetzen ist. Das Ergebnis einer Hashfunktion unterscheidet sich bereits bei geringfügigen Veränderungen im Programmcode. Das heißt, dass ein neuer Hashwert nicht eine neue Malware bedeutet, da die Funktionsweise des Schadcodes teilweise dieselbe bleibt. 
\newline Während für bekannte Schadprogramm-Varianten teilweise Detektionsmethoden existieren, sind neue Varianten unmittelbar nach ihrem Auftreten unter Umständen noch nicht als Schadprogramme erkennbar und daher besonders bedrohlich. 
\newline Zudem kann zwischen den Arten nicht sonderlich stark differenziert werden, da sie ineinander übergehen und ähnliche Methodiken und Ziele verfolgen.
\newpage 

   \subsubsubsection{Viren}
    \newline Viren, die Programme befallen, bestehen, wie ausführbare Programme, aus Code. 
    \newline Der spezifische Code ist stark von der jeweiligen Hardwareplattform und auch dem
    verwendeten Betriebssystem abhängig. Das ergibt sich mitunter daraus, dass Programme grundsätzlich einige Funktionen des Rechners nicht direkt ansprechen und durch System Calls bewilligt werden müssen.
    \newline Da sich der Virencode in den Programmcode einbetten muss, ist ein Virus meistens auf eine bestimmte Plattform und ein bestimmtes Betriebssystem als Wirt festgelegt und kann Systeme mit anderer Plattform oder anderem Betriebssystem nicht infizieren. 
    \newline Der ausführbare Code des Virus wird im Code des ursprünglichen Programms platziert und das Programm so modifiziert, dass beim Start zuerst der Virus aufgerufen wird.
    \newline Um länger unbemerkt zu bleiben, transferieren viele Viren danach die Kontrolle zurück an das ursprüngliche Programm, sodass dessen Funktion durch den Virus scheinbar nicht beeinträchtigt wird. Gelangt das Virus zur Ausführung, kann es weitere Programmdateien auf dem Rechner infizieren und gegebenenfalls weiteren Schadcode ausführen.
    
    Viren müssen nicht unbedingt nur ausführbare Programme infizieren. Einige Viren nisten sich
    im Hauptspeicher des Rechners ein und bleiben dort durchgehend aktiv. Solche Viren werden als \dq speicherresistente\dq{} Viren bezeichnet. 
    \newline In bestimmten Fällen können sich Viren auch über Dokumente, also eigentlich nicht ausführbare Daten, verbreiten. Dies ist unter anderem möglich, wenn die Anwendungsprogramme, welche die Dokumente aufrufen, Schwachstellen aufweisen,welche die Ausführung des injizierten Codes ermöglicht.
    \subsubsubsection{Trojaner}\label{Trojaner}
    \newline Trojaner sind  Schadsoftware-Varianten, über die meist destruktive oder datenstehlende Malware auf ein System geschleust wird. Anders als Viren und Würmer, sind Trojaner nicht in der Lage, sich selbstständig zu replizieren oder Dateien zu infizieren.
    Sie tarnen ihre Malware als nützliches Programm und hoffen darauf, dass arglose Nutzerinnen und Nutzer sie eigenhändig installieren. Täuschung ist hierbei ihre Verbreitungsstrategie. Häufig kommen Trojaner in fingierter Software vor, die von Angreifern manipuliert wurde. Diese Software ist zumeist als Download in unseriösen Quellen verfügbar.
    \newline In den meisten Fällen bestehen Trojaner aus zwei eigenständigen Programmen, die auf verschiedene Weise miteinander verknüpft sein können. Sogenannte Linker heften das Schadprogramm an eine ausführbare Wirtssoftware. Wird das vermeintlich nützliche Programm ausgeführt, startet gleichzeitig auch der Schadcode im Hintergrund.
    \newline Eine zweite Möglichkeit ist der Einsatz eines Droppers, der beim Start des Wirtsprogramms heimlich die Schadsoftware auf dem System ablegt. Während die Ausführung des schädlichen Programms im ersten Fall vom Wirt abhängig ist, kann es bei Einsatz des Droppers völlig unabhängig vom Trojaner agieren. 
    \newline Die dritte Möglichkeit ist die Integration des geheimen Codes in eine Wirtssoftware, wie es zum Beispiel bei vielen Browser-Plugins der Fall ist. Auch hier ist die Ausführung des schädlichen Programms an die Wirtssoftware gebunden. Wird diese beendet oder gelöscht, stehen auch die geheimen Funktionen nicht mehr zur Verfügung.
    \newline Weil der Trojaner in der Regel durch den Anwender selbst gestartet wird, hat er die gleichen Rechte wie der angemeldete Benutzer. Folglich kann er alle Aktionen ausführen, die auch der Nutzer ausführen könnte.
    
    \subsubsubsection{Würmer}
    \newline Bei einem Computerwurm handelt es sich um eine Malware, die sich selbstständig reproduziert und sich über Netzwerkverbindungen verbreitet.
    Der Computerwurm infiziert dabei normalerweise keine Computerdateien, sondern einen anderen Computer im Netzwerk. Dies geschieht, indem sich der Wurm repliziert. Diese Fähigkeit gibt der Wurm seinem Replikat weiter, wodurch auch dieser auf die gleiche Art und Weise andere Systeme infizieren kann. 
    \newline An der Stelle zeigt sich auch der Unterschied zwischen Computerwürmern und -viren. Computerwürmer sind eigenständige Programme, die sich selbst replizieren und im Hintergrund laufen, während Viren eine Host-Datei benötigen, die sie infizieren können. Aus diesem Grund kommt es häufig vor, dass ein Computerwurm erst bemerkt wird, wenn das Programm Systemressourcen verbraucht, wodurch andere Aufgaben verlangsamt oder angehalten werden.
    \subsubsubsection{Spyware/Adware}
    \newline Bei Spyware handelt es sich um eine Software, die ohne Wissen des Anwenders Aktivitäten auf dem Rechner oder im Internet ausspioniert und aufzeichnet. 
    Dabei werden Informationen weitergeleitet und durch verschiedene Methoden die Daten gesichert. Häufig wird hierbei das Keylogging verwendet um Benutzernamen, Passwörter und Bankdaten herauszufinden. Aufzeichnungen von Audio- und Videodaten sind ebenso wie das Erfassen von Inhalten aus E-Mail-, Messaging- und sozialen Apps keine Seltenheit. Jede Tätigkeit, die auf dem Rechner ausgeführt wird, ist nachvollziehbar. 
    Spyware kann nicht nur gefährlich sondern auch aufdringlich werden. Adware installiert sich selbst ebenso heimlich auf dem Rechner und spioniert den Browserverlauf aus, um mit passenden Anzeigen den User zu belästigen.
    Auch in der Gaming-Szene spielt Spyware eine immer größer einhergehende Rolle. Viele Programmierer lassen in der Installationkonsole eine Red-Shell-Spyware mitinstallieren, mit der die Online-Aktivität des Spielers verfolgt werden soll, um in Zukunft bessere Spiele zu veröffentlichen, die angepasster an den Verbraucher sind. Der Verbraucher wurde weder in Kenntnis davon gesetzt, noch hat dieser der Installation aktiv zugestimmt.
    Es handelt sich also demnach um Spyware.
    \subsubsubsection{Scareware}
    Scareware ist eine Art von Malware, die einen Virus oder ein anderes Problem auf einem Gerät zu erkennen vorgibt. Die Benutzer sollen zur Behebung des Problems schädliche Software downloaden oder kaufen. Üblicherweise ist Scareware die Vorstufe für einen Cyberangriff und nicht ein Angriff an sich.

    Scareware-Angriffe beginnen häufig mit einer Pop-up-Werbung, die den Eindruck erweckt, sie stamme von einer Sicherheitssoftware oder vom Betriebssystem des Rechners. Klicken die Benutzer darauf, werden sie auf eine infizierte Webseite geleitet. Auf dieser sollen Sie weitere Anweisungen zur Behebung ihres angeblichen Problems erhalten. Das umfasst beispielsweise die Installation eines neuen Tools oder Programms, einen Scan des Rechners, die Eingabe von Anmeldedaten, um mehr Informationen zu erhalten, oder das Hochladen von Kreditkartendaten für den Wiederherstellungsprozess. 
    Häufig führt das dazu, dass Benutzer unwissentlich schädliche Programme wie bereits erwähnte Malware-Arten auf das Gerät herunterladen.

    Scareware-Angriffe können auch über E-Mail erfolgen. Bei dieser Angriffsart verschicken die Angreifer E-Mails mit hoher Priorität oder Dringlichkeit, die Benutzer zum sofortigen Handeln auffordern. Die Links in der E-Mail täuschen den Benutzern vor, mit ihnen würden die Bedrohung behoben oder das System gescannt. Ein Klick darauf führt zum Download und zur Installation von infizierten Dateien, schädlichem Code oder Schadprogrammen.
    
    Ebenso wie viele andere Arten von Malware sind Scareware-Angriffe sehr problematisch, weil die Betrüger Zugang zu den Konto- oder Kreditkartendaten der Benutzer erlangen können, was Identitätsdiebstahl oder andere Betrugsarten ermöglicht.
    
    \subsubsubsection{Ransomware}
    Ransomware ist eine Art von Schadsoftware. \dq ransom \dq{} bedeutet übersetzt Lösegeld, was schon eine Aussage über die Funktionalität der Malware trifft. Beispielsweise spricht man von Ransomware, wenn der Schadcode darauf programmiert ist, Daten eines vermeindlichen Opfers so zu codieren, dass diese nicht wieder hergestellt werden können. Somit hat der Angreifer ein großes Druckmittel in der Hand, mit dem das Opfer zur Zahlung von Lösegeld aufgefordert werden kann. Bemerkbar macht sich Ransomware durch Erpresserbriefe oder einen blockierten Bildschirm, welcher erst wieder freigegeben wird, wenn die eingeforderte Summe gezahlt wird. So können mitunter große Unternehmen zu sehr hohen Zahlungen aufgefordert werden, da hier eine Codierung der Daten sehr schädigend für das Unternehmen werden kann. Die Zahlung garantiert jedoch nicht, dass der Computer und die Daten wieder freigegeben werden. Als Schutz vor Datenverlust gelten regelmäßige Backups der Daten und Updates der Sicherheitssysteme.
    \subsubsubsection{Root kit}
    Bei einem Rootkit handelt es sich nicht um eine einzelne Malware sondern um eine Sammlung verschiedener Schadprogramme, die sich über eine Sicherheitslücke in einen Computer einnistet und Angreifern den dauerhaften ferngesteuerten Zugriff (Remote-Access) auf diesen erlaubt. Wesentliches Merkmal von Rootkits ist, dass sie sich vor Virenscannern und Sicherheitslösungen verstecken können, sodass der Benutzer nichts von ihrer Existenz mitbekommt.

    Je nachdem, auf welcher Berechtigungsebene sich das Rootkit ausgebreitet hat, kann es dem Angreifer sogar umfassende Administrationsrechte verschaffen (in diesem Fall spricht man von einem Kernel-Mode-Rootkit), wodurch er die uneingeschränkte Kontrolle über den Rechner erhält.
    
    Sie bestanden anfangs zumeist aus modifizierten Versionen standardmäßiger Programme wie „ps“ (ein Unix command, der eine Liste aller aktiven Prozesse aufruft) und „passwd“ (zum Ändern des Benutzerpassworts). Daraus ergibt sich die Bezeichnung "rootkit": Mit „Root“ wird bei Unix der Administrator bezeichnet, der Wortteil „kit“ bedeutet so viel wie „Ausrüstung“ oder „Werkzeugkasten“. Der zusammengesetzte Begriff Rootkit umschreibt somit ein Set von Software-Werkzeugen, das einen Angreifer dazu ermächtigt, Root-Rechte über einen Computer zu erlangen (gemeint sind Kernel-Mode-Rootkits).

    Inzwischen existieren allerdings für eine Vielzahl von Betriebssystemen Rootkits. Auch für Windows- und andere Betriebssysteme ergibt die Bezeichnung „Rootkit“ durchaus Sinn: Denn einige Rootkits dringen bis in den Kernel, also den innersten Kern und damit die „Wurzel“ (Englisch: „root“) des Systems vor und werden von dort aus aktiv.
    
    Die Infiltration eines Systems durch ein Rootkit lässt sich allgemein in folgende Punkte gliedern:
    \begin{enumerate}
        \item Infektion des Systems
        \item Tarnung (Stealth)
        \item Einrichtung einer Hintertür (Backdoor)
    \end{enumerate}
    
    Infektion des Systems
    \newline Zunächst wird das Rootkit durch eine Sicherheitsücke oder einen Drive-By Download in den Computer eingenistet. An Passwörter und Zugangsdaten kommen die Angreifer in der Regel durch Beeinflussung oder bewusste Täuschung. Solche Sicherheitslücken sind demanch meist menschlicher Komponente. Je nach Berechtigungsebene, kann das Rootkit verschiedene Ziele verfolgen, die eher recht oberflächlich sind oder aber auch im direkten Kern agieren (Kernel-Mode-Rootkit).
    
    Tarnung (Stealth)
    \newline Sobald das Rootkit im System eingedrungen ist, verschleiert es seine Existenz. Dafür beginnt das Rootkit jene Prozesse zu manipulieren, über die Programme und Systemfunktionen Daten untereinander austauschen. Auf die Art erhält das Virenprogramm beispielsweise gefälschte Infomationen, aus denen sämtliche Hinweise auf das Rootkit herausgefiltert wurden.
    
    Einrichtung einer Hintertür (Backdoor)
    \newline Die Backdoor wird im folgenden dann als Mittel zum Zweck genutzt, damit der Angreifer mittels Remote-Access, ausgespähtem Passwort oder Shell in das System gelangt. Das Rootkit verschleiert dabei jegliche Hinweise und ermöglicht dadurch weitere Installationen wie beispielsweise Keylogger.
    
    \subsubsubsection{Backdoors}
    Eine Backdoor (deutsch \dq Hintertüren\dq) bezeichnet ein Teil einer Software, welche es dem Benutzer ermöglicht die Zugriffssicherung eines Systems zu umgehen und sich so Zugriff darauf zu verschaffen. Diese können bewusst vom Entwickler einer Software eingebaut sein. In diesem Fall können sie beispielsweise für die Reparatur von Systemen oder das Zurücksetzen von Adminpasswörtern verwendet werden. Jedoch kann eine Backdoor auch ungewollt auf ein System gelangen. Dabei wird über eine andere Malwareart eine Backdoor installiert. Dafür werden häufig Trojaner \ref{Trojaner} verwendet. Eine Backdoor ist immer nur ein Teil einer Malware und keine eigene Malware. Um die Funktionalität der Backdoor zu gewährleisten, werden also weitere Malwarekomponenten benötigt. Als Beispiel dafür sind Widgets oder kleinere Programme zu benennen, die auf dem Zielsystem installiert sind und vermeintlich harmlos wirken. Dennoch können genau diese den ferngesteuerten Zugriff über eine Backdoor ermöglichen. Werden diese Programme beendet oder deinstalliert, kann auch die Backdoor nicht aufrecht erhalten werden und somit kann auch kein weiterer Fernzugriff stattfinden.
    Die Nutzung von Backdoors ist unterschiedlich. Häufig wird eine Backdoor zum Datenklau oder für die Installation weiterer Malware verwendet. Auch die Ausspähung des Zielsystems oder auch die Verschlüsselung einzelner auf dem Ziel befindlicher Daten kann hier als Ziel genannt werden.
    
    \subsubsubsection{Krypto-Miner}
    Krypto-Miner sind eine Art von Malware, die darauf abzielt Rechenleistung des Opfers zu verwenden. Diese Malware-Art hat sich durch den Trend der Kryptowährungen verbreitet. Sie gewinnen immer mehr an Wert und sind für viele eine Investitionsmöglichkeit. Kryptowährungen können jedoch nicht nur mit Geld erworben werden, sondern auch mit der Bereitstellung von Rechenleistung. Übernimmt der PC eines Kryptominers die Rechnung einer Transaktion einer bestimmten Kryptowährung, wird der Besitzer mit der entsprechenden Währung dafür bezahlt. Dies ist dadurch begründet, dass er kurzfristig Rechenleistung verleiht. \newline
    Da die Rechenleistung, die benötigt wird um Kryptowährungen zu verdienen, jedoch immer teurer wird entsteht hier jedoch das Problem, dass das \dq schürfen\dq{} von Kryptowährungen nicht mehr effizient ist. Dieser Umstand kann mit Krypto-Mining-Viren umgangen werden. Hierbei wird über einen Virus die Rechenleistung des Ziel-Rechners zum schürfen der digitalen Währung verwendet. Den Ertrag daraus wird an die \it Krypto-Wallet \rm des Angreifers überwiesen. Somit können Opfer-PCs sehr stark verlangsamt werden, da die Rechenleistung zum schürfen der Kryptowährungen verwendet wird. Wichtig hierbei ist jedoch, dass der Nutzen eines solchen Virus davon abhängt, wie viel die zu schürfende Währung Wert ist. Denn umso wertvoller die Kryptowährung ist, desto mehr Rechenleistung wird zum gewinnen dieser benötigt. Das heißt, dass beispielsweise Bitcoin nicht über solche Viren gewonnen werden kann, da die benötigte Rechenleistung dafür bei den meisten PCs nicht ausreicht.
\end{otherlanguage}