\begin{otherlanguage}{ngerman}
\subsection{Malware}
\subsubsection{Was ist Malware und wie funktioniert sie?}

Laut Defintion ist Malware eine "bösartige Software"
(auf Englisch "malicious software").
\newline Allgemein dient der Begriff Malware als Klassifizierung von Dateien oder Software, die Schäden verursachen, sobald sie sich im System des Benutzers befinden. 
\newline Die häufigsten Arten sind:
\begin{enumerate}
    \item Viren
    \item Trojaner
    \item Würmer
    \item Spyware/Adware
    \item Scareware
    \item Ransomware
    \item Root kit und Backdoors
\end{enumerate}
Dabei können sowohl persönliche Schäden als auch Sachschäden entstehen.

Um einen solchen Malware Angriff zu tätigen, benötigt man einen Schadcode. 
Mithilfe dessen verschaffen sich Cyberkriminelle Zugriff auf das System des Opfers und klauen Passwörter oder andere sensible Daten.
Um den Datendiebstahl durchzuführen, muss ein Cyberkrimineller dafür sorgen, dass der Schadcode auf dem Zielsystem ausgeführt wird. 
Damit das klappt, betten diese den schädlichen Code in eine Datei ein, die das Opfer öffnen soll.
Nach der Durchführung erfüllt der Schadcode seinen Zweck und installiert zum Beispiel eine Backdoor oder startet einen Key-Logger. Key Logger können unter anderem die Eingabe von Passwörtern protokollieren.

\subsubsection{Arten von Malware}
Neue Schadprogramm Varianten entstehen, wenn die Funktionsweise der neuen Malware grundlegend von bereits vorhandenen Schadprogrammen abweicht. Wichtig ist hierbei, dass der Hashwert einzigartig ist, aber ein neuer Hashwert nicht mit einer neuen Malware gleichzusetzen ist. Das Ergebnis einer Hashfunktion unterscheidet sich bereits bei geringfügigen Veränderungen im Programmcode. Das heißt, dass ein neuer Hashwert nicht eine neue Malware bedeutet, da die Funktionsweise des Schadcodes teilweise die selbe bleibt. 
\newline Während für bekannte Schadprogramm-Varianten teilweise Detektionsmethoden existieren, sind neue Varianten unmittelbar nach ihrem Auftreten unter Umständen noch nicht als Schadprogramme erkennbar und daher besonders bedrohlich. 
\newline Zudem kann zwischen den Arten nicht sonderlich stark differenziert werden, da sie ineinander übergehen und ähnliche Methodiken und Ziele verfolgen.
\newpage 

   \subsubsubsection{Viren}
    \newline Viren, die Programme befallen, bestehen wie ausführbare Programme aus Code. 
    \newline Der spezifische Code ist stark von der jeweiligen Hardwareplattform und auch dem
    verwendeten Betriebssystem abhängig. Das ergibt sich mitunter daraus, dass Programme grundsätzlich einige Funktionen des Rechners nicht direkt ansprechen und durch System Calls bewilligt werden müssen.
    \newline Da sich der Virencode in den Programmcode einbetten muss, ist ein Virus meistens auf eine bestimmte Plattform und Betriebssystem als Wirt festgelegt und kann Systeme mit anderer Plattform oder anderem Betriebssystem nicht infizieren. 
    \newline Der ausführbare Code des Virus wird im Code des ursprünglichen Programms platziert und das Programm so modifiziert, dass beim Start zuerst der Virus aufgerufen wird.
    \newline Um länger unbemerkt zu bleiben, transferieren viele Viren danach die Kontrolle zurück an das ursprüngliche Programm, so dass dessen Funktion durch den Virus scheinbar nicht beeinträchtigt wird. Gelangt das Virus zur Ausführung, kann es weitere Programmdateien auf dem Rechner infizieren und gegebenenfalls weiteren Schadcode ausführen.
    
    Viren müssen nicht unbedingt nur ausführbare Programme infizieren. Einige Viren nisten sich
    im Hauptspeicher des Rechners ein und bleiben dort durchgehend aktiv. Solche Viren werden als \dq speicherresistente\dq{} Viren bezeichnet. 
    \newline In bestimmten Fällen können sich Viren auch über Dokumente, also eigentlich nicht ausführbare Daten, verbreiten. Dies ist unter anderem möglich, wenn die Anwendungsprogramme, welche die Dokumente aufrufen, Schwachstellen aufweisen,welche die Ausführung des injizierten Codes ermöglicht.
    \subsubsubsection{Trojaner}
    \newline Trojaner sind  Schadsoftware-Varianten, über die meist destruktive oder datenstehlende Malware auf ein System geschleust wird. Anders als Viren und Würmer, sind Trojaner nicht in der Lage, sich selbstständig zu replizieren oder Dateien zu infizieren.
    Sie tarnen ihre Malware als nützliches Programm und hoffen darauf, dass arglose Nutzerinnen und Nutzer sie eigenhändig installieren. Täuschung ist hierbei ihre Verbreitungsstrategie. Häufig kommen Trojaner in fingierter Software vor, die von Cyber-Kriminellen manipuliert wurde. Diese Software ist zumeist als Download in unseriösen Quellen verfügbar.
    \newline In den meisten Fällen bestehen Trojaner aus zwei eigenständigen Programmen, die auf verschiedene Weise miteinander verknüpft sein können. Sogenannte Linker heften das Schadprogramm an eine ausführbare Wirtssoftware. Wird das vermeintlich nützliche Programm ausgeführt, startet gleichzeitig auch der Schadcode im Hintergrund.
    \newline Eine zweite Möglichkeit ist der Einsatz eines Droppers, der beim Start des Wirtsprogramms heimlich die Schadsoftware auf dem System ablegt. Während die Ausführung des schädlichen Programms im ersten Fall vom Wirt abhängig ist, kann es bei Einsatz des Droppers völlig unabhängig vom Trojanischen Pferd agieren. 
    \newline Die dritte Möglichkeit ist die Integration des geheimen Codes in eine Wirtssoftware, wie es zum Beispiel bei vielen Browser-Plugins der Fall ist. Auch hier ist die Ausführung des schädlichen Programms an die Wirtssoftware gebunden. Wird diese beendet oder gelöscht, stehen auch die geheimen Funktionen nicht mehr zur Verfügung.
    \newline Weil der Trojaner in der Regel durch den Anwender selbst gestartet wird, hat er die gleichen Rechte wie der angemeldete Benutzer. Folglich kann er alle Aktionen ausführen, die auch der Nutzer ausführen könnte.
    
    \subsubsubsection{Würmer}
    \newline Bei einem Computerwurm handelt es sich um eine Malware, die sich selbstständig reproduziert und sich über Netzwerkverbindungen verbreitet.
    Der Computerwurm infiziert dabei normalerweise keine Computerdateien, sondern einen anderen Computer im Netzwerk. Dies geschieht, in dem sich der Wurm repliziert. Diese Fähigkeit gibt der Wurm seinem Replikat weiter, wodurch auch dieser auf die gleiche Art und Weise andere Systeme infizieren kann. 
    \newline An der Stelle zeigt sich auch der Unterschied zwischen Computerwürmern und -viren. Computerwürmer sind eigenständige Programme, die sich selbst replizieren und im Hintergrund laufen, während Viren eine Host-Datei benötigen, die sie infizieren können. Aus diesem Grund kommt es häufig vor, dass ein Computerwurm erst bemerkt wird, wenn das Programm Systemressourcen verbraucht, wodurch andere Aufgaben verlangsamt oder angehalten werden.
    \subsubsubsection{Spyware/Adware}
    \newline Bei Spyware handelt es sich um eine Software, die ohne Wissen des Anwenders Aktivitäten auf dem Rechner oder im Internet ausspioniert und aufzeichnet. 
    Dabei werden Informationen weitergeleitet und durch verschiedene Methoden die Daten gesichert. Häufig wird hierbei das Keylogging verwendet um Benutzernamen, Passwörter und Bankdaten herauszufinden. Aufzeichnungen von Audio- und Videodaten sind ebenso wie das Erfassen von Inhalten aus E-Mail-, Messaging- und sozialen Apps keine Seltenheit. Jede Tätigkeit, die auf dem Rechner ausgeführt wird, ist nachvollziehbar. 
    Spyware kann nicht nur gefährlich sondern auch aufdringlich werden. Adware installiert sich selbst ebenso heimlich auf dem Rechner und spioniert den Browserverlauf aus um mit passenden Anzeigen den User zu belästigen.
    Auch in der Gaming-Szene spielt Spyware eine immer größer einhergehende Rolle. Viele Programmierer lassen in der Installationkonsole eine Red-Shell-Spyware mitinstallieren mit der die Online-Aktivität des Spielers verfolgt werden soll, um in Zukunft bessere Spiele zu veröffentlichen, die angepasster an den Verbraucher sind. Der Verbraucher wurde weder in Kenntnis davon gesetzt, noch hat dieser der Installation zugestimmt.
    Es handelt sich also demnach um Spyware.
    \subsubsubsection{Scareware}
    \subsubsubsection{Ransomware}
    \subsubsubsection{Root kit und Backdoors}
   
\end{otherlanguage}