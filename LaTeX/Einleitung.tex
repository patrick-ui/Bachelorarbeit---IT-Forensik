\section{Einleitung} 

\subsection{Thema der Arbeit und Motivation}
In dieser Arbeit geht es um die Nutzung virtueller Umgebungen für die Analyse von schädlicher Software in einem automatisierten Verfahren. Dabei entstehen im Bereich der virtuellen Umgebung besonders komplexe Probleme, wie die Auswahl der Tools, des Betriebssystems oder des Cloud-Anbieters. Außerdem wird die Frage aufgeworfen, wie eine virtuelle Umgebung automatisiert und flexibel erstellt werden kann. Dabei ist die Unabhängigkeit vom Anbieter besonders wichtig, da das Malware-Analyse-Labor so flexibel sein soll, dass es bei vielen verschiedenen Anbietern mit wenig Aufwand konfiguriert und verwendet werden kann. 
\newline 
Wenn diese Umgebung erstellt wurde, ist es wichtig, dass diese sicher ist, denn Malware sollte nicht auf Rechnern mit sensiblen Daten Zugriff haben. Das heißt, dass die Art, wie die virtuelle Umgebung nach außen abgesichert ist, eine grundlegende Rolle für die Funktion dieser spielt. Allgemein soll durch diese Arbeit ein Mittel für die IT-Sicherheit geboten werden, um diese zu verbessern. Daher ist die Absicherung einer der wichtigsten Punkte. Zu diesen wichtigen Punkten zählt unter anderem auch die Auswahl der Tools. Diese legt fest, wie die Umgebung später genutzt wird.
\newline
Weitere Punkte, die außerdem aufgegriffen werden, sind die Konfiguration der Kommunikationsmöglichkeiten, die der Analyse-Umgebung zur Verfügung stehen. Um hier feste Aussagen treffen zu können, muss die allgemeine Funktionsweise verschiedener Malwarearten durchblickt werden.
\subsubsection{Relevanz}
Malware-Analyse gibt Auskunft über das Verhalten von der untersuchten Schadsoftware. Die Informationen, die dabei gewonnen werden, können für die Sicherheitsmaßnahmen des Zielsystems wichtige Aufschlüsse darüber geben, wie das Eindringen dieser Malware-Art in das System in Zukunft unterbunden werden kann. Zudem kann Anhand der Analyse der Malware der Schaden, den diese anrichtet, abgeschätzt werden. Auch hilft die Untersuchung bestimmter Schadcodes dabei einen IoC-Katalog zu führen, der in die Sicherheitskomponenten eines Netzwerks eingepflegt werden kann. So wird Malware kategorisch ausgeschlossen.\footcite[S.1]{WhyAnaly} 
\newline Außerdem kann die Funktionalität der Malware unterbunden werden, wenn bekannt ist, mit welchen Ressourcen die Malware arbeiten muss, um zu funktionieren. Alles in Allem bietet Malware-Analyse eine große Bereicherung für die IT-Sicherheit.\footcite{BookPrcMal} Das Problem dabei ist oft, dass die notwendige Umgebung für die Analyse von Malware nicht überall vorhanden ist. Es besteht die Möglichkeit des Mietens solcher Umgebungen bei externen Firmen\footcite{MietUmg}, was aber teilweise sehr kostspielig ist. Die Grundlage für eine Alternative dazu soll in dieser Arbeit geschaffen werden.
\subsection{Forschungsstand}
Malware-Analyse und die Erstellung dieser ist ein Thema mit hohem Stellenwert in der IT. Der Forschungsstand zu einzelnen Malwarearten, sowie zu der Malware-Analyse, ist ein dementsprechend gut erforschtes Themengebiet mit vielen wissenschaftlichen Publikationen. Auch \dq Terraform \dq{} als Provisioning-Tool für Cloud-Umgebungen ist ein geläufiges Werkzeug. Es finden sich daher auch hierzu wissenschaftliche Artikel und Publikationen, die den Umgang und die Möglichkeiten des Tools betrachten. 
\newline
In dieser Arbeit sollen diese beiden Parts zusammengefügt werden, denn bei der Recherche ist auffällig geworden, dass die beiden Themen nur separat voneinander untersucht wurden. Daher sollen sie hier verbunden werden, um die Möglichkeiten aufzuzeigen, die Terraform in der Welt der Malware-Analyse bietet. Dazu liegen zum jetzigen Zeitpunkt keine Arbeiten und Publikationen vor.
\subsection{Aufbau der Arbeit}
Zunächst wird im zweiten Kapitel eine theoretische Grundlage geschaffen. Diese ist untergliedert in Grundlagen zu \dq Terraform\dq, zu Cloud Computing, zu virtuellen Maschinen und zu Malware, sowie ihrer Analyse. 
\newline Anschließend befasst sich Kapitel drei mit dem praktischen Teil dieser Arbeit. Dafür wird das Vorgehen mit den einzelnen Werkzeugen, um das Ziel zu erreichen, beschrieben. Hierbei zeigt der Teil Terraform \ref{Praktisches Terraform} auf, wie die Konfigurationsdateien aussehen. Im Anschluss dazu geht es um die virtuelle Umgebung, die aus den entsprechenden Konfigurationsdateien hervorgeht. Die Tools, die hier installiert sind, werden im nächsten Punkt genauer erläutert. Hier wird einerseits auf die einzelnen Tools eingegangen, andererseits ist hier auch der Zweck dieser Tools erklärt.
\newline
Im vierten Kapitel werden die Ergebnisse dargestellt. Dazu wird die tatsächliche Vorgehensweise dargelegt und die Verwendung, der in Kapitel drei genannten Methoden, aufgezeigt. Außerdem werden hier die Ergebnisse und deren Aussagekraft über die Forschung eingeordnet. 
\newline
Das fünfte Kapitel ist die Diskussion, in welcher die Ergebnisse genauer interpretiert und bewertet werden. Auch mögliche Verbesserungsvorschläge sind hier aufzugreifen.
\newline
Als Abschluss dieser Arbeit steht das Fazit im sechsten Kapitel. Dort werden die wichtigsten Erkenntnisse, aus der Forschung, im Rahmen der Arbeit dargelegt. Anschließend folgt das Literaturverzeichnis, sowie der Anhang.
\subsection{Ziel der Arbeit}
Das konkrete Ziel dieser Arbeit ist es, die Möglichkeit zu bieten, mithilfe bestimmter Tools eine virtuelle Umgebung für die Analyse von Malware zu erstellen. Der Fokus liegt darauf, dass die Flexibilität dieser Umgebung geboten wird. Außerdem sollen für jede Analysetechnik Tools zur Verfügung stehen. Ein weiterer Faktor, der beachtet wird, ist, dass die Kosten für diese Umgebung sehr günstig im Vergleich zu den kommerziellen Lösungen sind. So ist das Ziel, jedem, der Malware analysieren möchte oder muss, eine kostengünstige, erste Grundlage bieten zu können, um die Analyse durchführen zu können.