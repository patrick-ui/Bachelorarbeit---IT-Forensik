
\begin{otherlanguage}{ngerman}
\subsection{Virtuelle Maschinen}
\subsubsection{Was ist eine virtuelle Maschine?}
Virtuelle Maschinen bilden mit Infrastructure as a Service (IaaS) eines der drei Servicemodelle des Cloud Computings. Sie beruhen auf der Idee Hardware, Rechenleistung, Speicherplatz und Netzwerkressourcen aus der Cloud bereitzustellen. So hat der Benutzer von virtuellen Maschinen das gleiche Erlebnis wie er es auch bei physischen Geräten hat. 
\newline 


\subsubsection{Erstellen einer virtuellen Maschine}
Eine virtuelle Maschine kann in der Regel bei einem \it Cloud Provider \rm erstellt werden. Hierfür kann die grafische Umgebung der jeweiligen Provider genutzt werden. Bevor jedoch eine VM erstellt wird muss darüber nachgedacht werden zu welchem Zweck die VM ist. Daraus ergeben sich wichtige Konfigurationsmerkmale wie das Betriebssystem oder die Sicherheitskonfigurationen. 
\newline
-Terraform

\subsubsection{Absicherung einer virtuellen Maschine}
Die Sicherung der virtuellen Maschine kann je nach Einsatzgebiet eine große Bedeutung haben. Um Sicherheitskonfigurationen einzubinden kann die Firewall bearbeitet werden. Auch dafür bietet Terraform eine Möglichkeit. 
\newline 
-Terraform
\end{otherlanguage}