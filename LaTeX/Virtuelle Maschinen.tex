\begin{otherlanguage}{ngerman}
\subsection{Virtuelle Maschinen}
\subsubsection{Was ist eine virtuelle Maschine?}
Virtuelle Maschinen bilden mit Infrastructure as a Service (IaaS) eines der drei Servicemodelle des Cloud Computings. Sie beruhen auf der Idee Hardware, Rechenleistung, Speicherplatz und Netzwerkressourcen aus der Cloud bereitzustellen. So soll der Benutzer von virtuellen Maschinen das gleiche Erlebnis wie er es auch bei physischen Geräten haben. 
\newline 
\newline
Der Vorteil daran, dass virtuelle Maschinen ein aus dem Internet bereitgestellter Dienst ist ist die Flexibilität. Durch die Verwendung von virtuellen Maschinen können Kosten gespart werden. Das rührt daher, dass der Kunde sein System den Anforderungen entsprechend planen und umsetzen kann. Ist die VM einmal aufgesetzt und soll vergrößert werden, muss nicht die Hardware ausgetauscht werden. Stattdessen wird die Größe der virtuellen Umgebung individuell angepasst. Damit einher geht auch die Änderung der Kosten für das System. 
\newline 
Zudem ist die Unabhängigkeit vom Host-System ein Vorteil. In vielen unternehmen wird Windows als graphisches Betriebssystem benutzt. Dennoch kann es sein, dass ein Rechner mit einem Linux-System für manche Prozesse besser geeignet ist. Alle Rechner auf Linux umzustellen nur, weil bestimmte Abläufe so besser funktionieren ist aber auch keine Lösung. Auch die Anschaffung eines einzelnen Rechners, um Linux zu benutzen ist nicht effizient und praktisch. Stellvertretend dafür kann Linux auf einer virtuellen Umgebung laufen. Die Kosten dafür sind je nach Gerätkonfiguration weit unter dem Hardwarepreis für ein Gerät mit gleicher Ausstattung. Zudem ist die Verfügbarkeit des Gerätes gewährleistet, da theoretisch von überall wo es freien Internetzugang gibt auf die Cloud zugegriffen werden kann. 



\subsubsection{Erstellen einer virtuellen Maschine}
Eine virtuelle Maschine kann in der Regel bei einem \it Cloud Provider \rm erstellt werden.  Bevor jedoch eine VM erstellt wird muss darüber nachgedacht werden zu welchem Zweck die VM ist. Daraus ergeben sich wichtige Konfigurationsmerkmale wie das Betriebssystem oder die Sicherheitskonfigurationen. Nach dieser Planungsphase kommt es zum erstellen der VM. Hierfür kann die grafische Umgebung der jeweiligen Provider genutzt werden. Das Erstellen eines einfachen Servers ist so schnell und unkompliziert gemacht. Für die Verbildlichung einer solchen Weboberfläche wird in dieser Arbeit die \it Hetzner-Cloud \rm verwendet. Hier wird zuerst über das Feld \dq NEUES PROJEKT\dq{} ein Projekt erstellt. In diesem Projekt kann mithilfe des Buttons \dq SERVER HINZUFÜGEN\dq{} ein Server erstellt werden. Der Server hat die Konfigurationsoptionen:
\begin{acronym}
    \acro{Standort}{- bestimmt woher die Serverleistungen kommen sollen}
    \acro{Image}{- ist in die Reiter \it OS-Images \rm und \it Apps \rm aufgeteilt. Es bietet die Möglichkeit das Betriebssystem und Tools zu installieren.} 
    \acro{Typ}{- hier kann die Rechenleistung, Festplattengröße und das maximale Datenaufkommen festgesetzt werden. Der Preis der jeweiligen Cloudlösung ist mit aufgelistet.}
    \acro{Volume}{- hier kann SSD-Speicher festgelegt werden}
    \acro{Networking}{- Festlegen über welche Netze der Server kommunizieren darf}
    \acro{Firewalls}{- können hinzugefügt werden, nachdem sie in einem anderen Teil der\it Hetzner Cloud \rm erstellt wurden}
    \acro{Zusätzliche Features}{- Festlegung von Benutzerdaten, Backups und Platzierungsgruppen}
    \acro{SSH-Key}{- hier kann der SSH-Key hinzugefügt werden}
    \acro{Name}{- Festlegung des Servernamens}
\end{acronym}
\newline
\newline
Die virtuelle Maschine kann auch mit Terraform aufgesetzt werden. Das macht die Konfiguration besser an andere Provider anpassbar. Dafür müssen in den Konfigurationsdateien von Terraform nur wenige resourcen geändert werden. Wie eine VM mit Terraform aufgesetzt werden kann, wird in einem späteren Teil der Arbeit aufgegriffen(\ref{Praktisches Terraform}).

\subsubsection{Absicherung einer virtuellen Maschine}
Die Sicherung der virtuellen Maschine kann je nach Einsatzgebiet eine große Bedeutung haben. Um Sicherheitskonfigurationen einzubinden kann die Firewall bearbeitet werden. Auch dafür bietet Terraform eine Möglichkeit. 
\newline 
-Terraform
\end{otherlanguage}