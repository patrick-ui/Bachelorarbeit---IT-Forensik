\begin{otherlanguage}{ngerman}
\section{Diskussion}
\subsection{Analyse der Ergebnisse}

\subsection{Bewertung der Ergebnisse} \label{Bewertung}
Die Funktion der Cloud-Umgebung als Malware-Analyse-Umgebung wurde erfüllt. Alle notwendigen Tools für die statische und die dynamische Analyse sind installiert, sodass die Malware auf dieser Umgebung problemlos analysiert werden kann. Ein Problem, was bei der Analyse von Malware jedoch auftreten könnte ist, dass Malware oft erkennt ob sie in einer virtuellen Umgebung ist. Das führt dazu, dass sich der Schadcode bei der dynamischen Analyse anders verhält, als er es in der Realität tun würde. In der Malware wird ein solches verhalten durch bestimmte Punkte ausgelöst. Dazu zählt, dass geprüft wird welche Ports für die kommunikation mit anderen Systemen offen sind. Diese weichen bei virtuellen Analyseumgebungen meist stark vom Normalfall ab. So wird der Malware vermittelt, dass sie sich in einer abgesicherten Umgebung befindet, was teilweise ein anderes Verhalten des Programms zur Folge hat. Außerdem kann die Malware an Art und Speicherort der Dateien, die auf dem System vorhanden sind, eine virtuelle Maschine identifizieren. So etwas wird oft in Schadsoftwares verwendet um die Analyse zu erschweren oder auch unmöglich zu machen. Hierbei kommt es jedoch immer auf die spezifische Malware an, wodurch keine allgemeine Aussage getroffen werden kann. Dies ist ein Punkt der negativ zu bewerten ist.\newline
Positiv anzumerken ist jedoch, dass eine flexible Cloud-Umgebung die theoretisch bei jedem, mithilfe des GitHub-Ordners, erstellt werden kann, Vorteile bietet. Dies ist besonders für Firmen ein Grund auf eigene Lösungen für die Analyse von Schadcode umzusteigen. Schickt man Malware, die auf dem internen System gefunden wurde an Dritte zur Analyse, so können auch firmeninterne Daten an diese Dritten gelangen. Darunter zählt beispielsweise, dass die virtuelle Umgebung die der externe Dienstleister aufbaut möglichst ähnlich mit der internen Umgebung sein muss. Der Netzwerkaufbau ist in den meisten Firmen jedoch vertraulich und sollte dementsprechend nicht an Externe gelangen. Mithilfe dieser Arbeit bietet sich nun aber die Möglichkeit eine virtuelle Umgebung selbst zu erstellen und anzupassen ohne, dass dafür zusätzliche Dienstleistungen in Anspruch genommen werden müssen. 
Festzustellen ist , dass die Funktionsweise der virtuellen Umgebung als Analyselabor trotzdem gegeben ist.
\subsection{Ausblick}
Eine angemessene Benutzerfreundlichkeit ist ein wichtiger Punkt dieser Arbeit. Eine virtuelle Umgebung ist ohne Benutzerhandbuch teilweise schwer zu durchblicken und daher nicht wirklich benutzerfreundlich. Um dem Benutzer dabei etwas an die Hand zu geben ist die Erstellung eines \it Wikis\rm. Dieses soll als Benutzerhandbuch dienen und die wichtigsten Bereiche und Hinweise zur Erstellung und Verwendung der virtuellen Umgebung abdecken. Zu jeder Funktionalität sollen hier Begründungen und Hinweise zu finden sein. Außerdem werden an manchen Punkten Alternativen oder Verbesserungsmöglichkeiten aufgezeigt werden. \newline
Auch der in der Bewertung(\ref{Bewertung}) angesprochene Punkt ist eine Aufgabe die in Zukunft behoben werden soll. So soll die Möglichkeit, jegliche Malware mit dieser Umgebung zu analysieren geboten werden.
\end{otherlanguage}