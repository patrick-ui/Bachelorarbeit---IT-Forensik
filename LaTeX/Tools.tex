\begin{otherlanguage}{ngerman}
\subsection{Tools}
Für eine ausführliche Analyse der vorliegenden Malware werden Tools benötigt. Ziel ist hierbei, dass die Tools die statische und dynamische Analyse abdecken. So kann die Malware in der Umgebung vollumfänglich untersucht werden. Für die Analyse von Malware in Linux-Systemen gibt es bereits viele Möglichkeiten. Betriebssysteme wie \dq Kali-Linux\footnote{} \dq{} oder auch \dq Remnux\footnote{} \dq{} sind Linux-Varianten, die speziell für die Analyse von schädlichen Programmen ausgelegt sind. Hier sind bereits viele Tools vorinstalliert. Obwohl die Arbeit mit den eben genannten Betriebssystemen die Auswahl der Tools vereinfachen würde, werden diese hier nicht verwendet. Viele Cloud-Provider bieten für virtuelle Server diese Betriebssysteme nämlich nicht an. Stattdessen wird auf den hier erstellten virtuellen Servern Ubuntu als Linuxdistribution verwendet. Dieses ist bei vielen Cloud-Anbietern verfügbar und unterstützt somit das Ziel der Flexibilität der Umgebung. 
\newline
Als erstes geht es um die Tools für die statische Analyse. \dq TrID \dq{} ist zum Identifizieren von Dateien sehr nützlich. Die Daten, mit denen festgestellt wird, um was für eine Art von Datei es sich handelt werden aus einer Datenbank bezogen, die mit dem Programm verknüpft ist. Diese wächst immer weiter, sodass hier nicht mit veralteten Datenbanken gearbeitet wird. Das sichert die Verlässlichkeit der Daten. Zum Zeitpunkt der Recherche wird dieses Tool immer noch regelmäßig aktualisiert, was die Aktualität der Datenbanken unterstreicht.\footcite{TrID} 
\newline 
Für die erweiterte statische Analyse werden Disassembler und Decompiler verwendet. Diese helfen dabei, die Funktionsweise der schädlichen Datei zu verstehen. Dafür wird die Maschinensprache die der Computer interpretieren kann in eine für Menschen lesbare Sprache umgewandelt. Das Programm \dq Ghidra \dq{} ist für diesen Zweck sehr geeignet. Ghidra wurde ursprünglich für den Geheimdienst der USA entwickelt und ist nun als Open-Source-Software erhältlich. Es interpretiert die Maschinensprache und erzeugt aus einem ausführbaren Programm einen Code um, der für Menschen verständlich ist. So kann die Vorgehensweise des zu untersuchenden Programms analysiert und verstanden werden.\footcite{}
\newline
Als Debugger wird in dieser Cloud-Umgebung \dq GNU Debugger (GDB) \dq{} benutzt. Dies ist ein weiteres Tool, welches in einer Linuxumgebung für die Analyse von Schadcode verwendet werden kann. Es wird für die dynamische Analyse verwendet. \dq GDB \dq{} wird dafür verwendet, Programme zu starten und festzustellen, was das verhalten des gestarteten Programms beeinflussen kann. Außerdem können Bedingungen für den Start und Stop des Programmes festgelegt werden. Zudem wird dokumentiert was das Programm gemacht hat, sobald es gestoppt wurde.\footcite{}
\newline


\subsection{Wie die Malware auf die virtuelle Umgebung gelangt}
Die virtuelle Umgebung soll zur gezielten Analyse von Malware-Samples verwendet werden. Anders als bei \it Honeypods \rm, die Malware \dq anlocken \dq{}, gelangt die Schadsoftware also nicht zufällig auf das System, sondern muss gezielt heruntergeladen werden. 
\newline
Zuerst stellt sich die Frage, woher man die Malware bekommt. Wird sie einem zugeschickt mit der Aufgabe sie zu untersuchen ist sie meist bereits verpackt und über einen Download-Link abrufbar. Wird die Malware auf einem System gefunden und man möchte sie untersuchen, sollte man die zum Schadcode gehörenden Dateien verpacken. Dazu eignet sich das Dateiformat \dq Tar \dq{} sehr gut. Bei Dateien mit der Endung \dq .tar \dq{} werden alle zu verpackenden Dateien in einer Datei zusammengeführt. Der Inhalt dieser Datei ist erst nach dem entpacken wieder sichtbar. 
\newline
Wenn das Malware-Sample verpackt ist, bietet es sich an die Malware über einen Download-Link bereitzustellen. In dieser Arbeit ist diese Plattform \dq Seafile \dq. Auf der Cloud-Umgebung ist die Malware anschließend über das Kommando: \newline \tt wget [DOWNLOAD-LINK]\rm \newline herunterladbar. Dies kann nur erfolgen, wenn die Firewall ausgeschaltet ist, da in dieser die Kommunikation in das Internet verboten ist. Ist der verpackte Schadcode auf der Cloud-Umgebung, muss die Firewall wieder aktiviert werden. Anschließend kann die Malware für die Analyse entpackt werden. Dies geschieht unter Linux mit dem Kommando: \newline \tt tar -xvzf [NAMEDERDATEI]. tar. gz \rm \newline
Ab diesem Zeitpunkt ist die Malware auf der Analyseumgebung angekommen und ist bereit für die Analyse.
\newpage
\end{otherlanguage}