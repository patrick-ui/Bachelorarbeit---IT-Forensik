\newpage
\begin{otherlanguage}{ngerman}
\section*{Glossar}
\begin{acronym}
 \acro{Admin(istrator)}{- ist ursprünglich der Verwalter eines Netzwerkes. Er hat die meisten Rechte und darf auf alle Einstellungen zugreifen}
 \acro{App}{- ist eine Anwendung}
 \acro{Backdoor}{- bezeichnet einen Teil einer Software, der es dem User ermöglicht, unter Umgehung der normalen Zugriffssicherung Zugang zum Computer oder einer sonst geschützten Funktion eines Programms zu bekommen}
 \acro{Browser Plugin}{- sind kleine Zusatzprogramme, die innerhalb von Browsern die Inhalte und Funktionen erweitern.}
 \acro{Cloud}{- ist in der IT die Abkürzung für Cloud Computing und wird auch als Rechnerwolke oder Datenwolke bezeichnet}
 \acro{Cloud Computing}{- beinhaltet Technologien und Geschäftsmodelle um IT-Ressourcen dynamisch zur Verfügung zu stellen und ihre Nutzung nach flexiblen Bezahlmodellen abzurechnen}
 \acro{Cloud Provider}{- Anbieter für Cloud Computing}
 \acro{Cloud Server}{- ist ein virtueller Server. Dieser bezieht seine Hardwarekomponenten von externen Dienstleistern und wird über das Internet angeboten.}
 \acro{Download}{ - das Herunterladen von Software}
 \acro{Dropper}{}
 \acro{Firewall}{- ist ein Sicherungssystem um Netzwerke und Computer vor unerwünschten Zugriffen zu schützen}
 \acro{Hardware}{}
 \acro{HashiCorp}{}
 \acro{Hashwert}{}
 \acro{Host-System}{}
 \acro{Infrastructure-as-Code-Tool}{}
 \acro{Key-Logger}{}
 \acro{Linker}{}
 \acro{Malware}{}
 \acro{Netzwerkressourcen}{}
 \acro{Open-Source-Software}{- ist Software deren Quellcode der Öffentlichkeit zugänglich ist. Die Benutzung des Codes steht dem Benutzer selbst offen.}
 \acro{Oracle}{}
 \acro{OS-Images}{- ist eine komprimierte Sammlung von Referenzdateien und Ordnern, die zum Installieren und Konfigurieren eines neuen Betriebssystems auf einem Computer verwendet werden}
 \acro{Provisioning}{}
 \acro{Rechenleistung}{}
 \acro{Remote-Access}{}
 \acro{Rootkits}{}
 \acro{Schadprogramm (Code)}{}
 \acro{Shell}{}
 \acro{Software}{}
 \acro{Speicherplatz}{}
 \acro{System Calls}{}
 \acro{Ubuntu}{}
 \acro{Wiki}{- ist eine Sammlung von Informationen und Beiträgen im Internet zu einem bestimmten Thema, die von den Nutzern selbst bearbeitet werden können}
 \acro{Workflow}{- die Abwicklung arbeitsteiliger Vorgänge}
\end{acronym}
\end{otherlanguage}