\newpage
\begin{otherlanguage}{ngerman}
\section*{Glossar}
\begin{acronym}
 \acro{App}{- ist eine Anwendung}
 \acro{Cloud}{- ist in der IT die Abkürzung für Cloud Computing und wird auch als Rechnerwolke oder Datenwolke bezeichnet}
 \acro{Cloud Server}{- ist ein virtueller Server. Dieser bezieht seine Hardwarekomponenten von externen Dienstleistern und wird über das Internet angeboten.}
 \acro{Firewall}{- ist ein Sicherungssystem um Netzwerke und Computer vor unerewünschten Zugriffen zu schützen}
 \acro{Open-Source-Software}{- ist Software deren Quellcode der Öffentlichkeit zugänglich ist. Die Benutzung des Codes steht dem Benutzer selbst offen.}
 \acro{OS-Images}{- ist eine komprimierte Sammlung von Referenzdateien und Ordnern, die zum Installieren und Konfigurieren eines neuen Betriebssystems auf einem Computer verwendet werden}
 \acro{Workflow}{- die Abwicklung arbeitsteiliger Vorgänge}
 \acro{Cloud Provider}{- Anbieter für Cloud Computing}
\end{acronym}
\end{otherlanguage}