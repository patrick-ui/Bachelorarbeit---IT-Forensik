
\begin{otherlanguage}{ngerman}
\subsection{Cloud Computing}
Cloud Computing (\it deutsch Rechnerwolke oder Datenwolke \rm) ist das Bereitstellen von einzelnen Anwendungen bis hin zu Rechenzentren. Die Ressourcen hierfür werden aus der \it Cloud \rm genommen, welche auch als \it Internet \rm bekannt ist. Durch flexible Einstellungsoptionen ist es durch Cloud Computing möglich IT-Systeme bedarfsspezifisch und flexibel anzubieten. Dabei ist auch zu beachten, dass Cloud Computing viele Vorteile mit sich bringt. Darunter Fallen beispielsweise die Kosten und die Skalierbarkeit virtueller Systeme. 
\newline 
Cloud Computing wird in drei Arten aufgeteilt: Public (öffentlich), Privat (privat) und Hybrid (Mischung aus öffentlich und privat). Bei der öffentlichen Form besitzt ein Dienstleister die benötigten Ressourcen, welchen man \it Cloud Provider \rm nennt. Dabei werden die den Kosten entsprechenden Komponenten dem Kunden über das Internet zur Verfügung gestellt. Dabei spielen die Kosten als Vorteil eine große Rolle, da die Hardware nicht selbst angeschafft werden muss. Zudem sind Hardware-Wartungen nicht nötig, da diese vom \it Cloud Provider \rm übernommen werden. Durch die Vielzahl an Hardware-Komponenten, die große Anbieter haben, steigt die Skalierbarkeit sehr stark. Die Servermenge ist außerdem ausschlaggebend dafür, dass das Risiko eines Ausfalls minimiert wird. 
\newline
Die private Form des Cloud Computings unterscheidet sich von der öffentlichen Form. Besonders im Aspekt des Zugriffs auf die Hardware treten Unterschiede auf. Während bei der öffentlichen Form die selbe Hardware von verschiedenen Kunden genutzt wird, wird die Hardware bei privaten Cloud Computing-Lösungen nur einem Kunden zugewiesen. Diese kann hierbei im Besitz des Kunden sein und sich in seiner Organisation befinden. Auch kann sich die Hardware in Räumen von Drittanbietern befinden. Dabei ist sie dennoch nur für einen Käufer der Systemkomponenten bestimmt. So können Unternehmen ihre Netzwerkrichtlinien souverän bestimmen. Zudem gewährleistet die private Nutzung von Hardware mehr Sicherheit. 
\newline
In der hybriden Form sind die beiden vorangegangenden Modelle vereint. Ziel dabei ist es die Vorteile beider Formen in einer Form zu verschmelzen. Beispielsweise kann die große Flexibilität von öffentlichen Clouds benutzt werden. Währendessen stellt die private Cloud, durch ihre höhere Sicherheit, eine Zone für vertrauliche Objekte des Netzwerkes da. Aus der Zusammenführung der beiden Modelle ergibt sich eine Form, die die Vorteile vereint. Die Netzwerkkontrolle ist geboten, die Kosteneffizienz ist hoch und es wird ein hohes Maß an Flexibilität geboten. 

\subsubsection{Die Arten von Cloud-Diensten}
\bf Infrastructure as a Service (IaaS)\rm oder zu deutsch: \it Infrastruktur als Dienstleistung \rm stellt die notwendigen Systemkomponenten bereit. Dazu zählen beispielsweise der Arbeitsspeicher, die Festplatte oder der Prozessor. Diese Komponenten können von Drittanbietern oder von der Organisation selbst bereitgestellt werden. 
\newline 
\bf Platform as a service (PaaS) \rm  oder zu deutsch: \it Plattform als Dienstleistung \rm stellt die Platform für Entwicklungen und Anwendungen bereit. Diese wird von der folgenden Dienstleistung \it Software as a service (SaaS) \rm benötigt. Kunden für PaaS sind zum Beispiel Anwendungsentwickler. 
\newline 
\bf Software as a service (SaaS)) \rm befasst sich mit der Software von Systemen. Software lässt sich mit dem veränderbaren Part des Systems definieren (soft = weich, ware = Ware). Dabei werden Anwendungen, welche Online funktionieren angeboten. Diese sind in der Regel skalierbar, sodass sie auf Wünsche einzelner Organisationen eingehen können.
\end{otherlanguage}