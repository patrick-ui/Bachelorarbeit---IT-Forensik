\documentclass[12pt,oneside]{article}

%%%%%%%%%%%%%%%%%%%%%%%%%%%%
%%   Zusaetzliche Pakete  %%
%%%%%%%%%%%%%%%%%%%%%%%%%%%%
\usepackage{enumerate}  
\usepackage{fancyhdr}
\usepackage{a4wide}
\usepackage{graphicx}
\usepackage{palatino}
\usepackage{multirow}
\usepackage{booktabs}
\usepackage{titlesec}
\usepackage{acronym}% http://ctan.org/pkg/acronym
\usepackage{enumitem}% http://ctan.org/pkg/enumitem

%folgende Zeile auskommentieren für englische Arbeiten
\usepackage[ngerman]{babel}
%folgende Zeile auskommentieren für deutsche Arbeiten
%\usepackage[ngerman, english]{babel}

\usepackage[T1]{fontenc}
\usepackage[utf8]{inputenc}
\usepackage[bookmarks]{hyperref}
\usepackage[justification=centering]{caption}
\usepackage[style=authoryear,natbib=true,backend=biber,maxbibnames=20]{biblatex}
\usepackage{csquotes}
\bibliography{literatur}

\setlength{\parindent}{0em} 
\setlist[itemize]{noitemsep, topsep=0pt}
\setlist[enumerate]{noitemsep, topsep=0pt}

\newcommand{\subsubsubsection}[1]{\paragraph{#1}\mbox{}\\}
\setcounter{secnumdepth}{4}
\setcounter{tocdepth}{4}
%%%%%%%%%%%%%%%%%%%%%%%%%%%%%%
%% Definition der Kopfzeile %%
%%%%%%%%%%%%%%%%%%%%%%%%%%%%%%

\pagestyle{fancy}
\fancyhf{}
\cfoot{\thepage}
\setlength{\headheight}{16pt}

%%%%%%%%%%%%%%%%%%%%%%%%%%%%%%%%%%%%%%%%%%%%%%%%%%%%%
%%  Deckblatt (Platzhalter)  %%
%%%%%%%%%%%%%%%%%%%%%%%%%%%%%%%%%%%%%%%%%%%%%%%%%%%%%
\thispagestyle{empty}
%Platzhalter für späteres Deckblatt
\hspace{-2cm}\includegraphics{pabu5097_3677_cover.pdf}
\newpage

%%%%%%%%%%%%%%%%%%%%%%%%%%%%
%%  Beginn des Dokuments  %%
%%%%%%%%%%%%%%%%%%%%%%%%%%%%

\begin{document}

\lhead{}
\pagenumbering{Roman} 
    \setcounter{page}{1}

\tableofcontents
\clearpage

%%%%%%%%%%%%%%%%%%%%%%%%%%%%
%%  Kurzzusammenfassung   %%
%%%%%%%%%%%%%%%%%%%%%%%%%%%%
\lhead{Zusammenfassung}
%\section*{Zusammenfassung}
\addcontentsline{toc}{section}{Zusammenfassung}
\newpage
\newpage
\begin{otherlanguage}{ngerman}
\section*{Zusammenfassung}

In dieser Arbeit soll das Thema der forensischen Analyse von schädlicher Software behandelt werden. Explizit ist damit gemeint, dass eine Umgebung für diese Untersuchung in wenigen Sekunden geschaffen werden kann, welche den nötigen Sicherheitsstandards entspricht und so eine sichere Beobachtung des Malwareverhaltens gewährleisten kann. Diese Umgebung soll unabhängig vom Anbieter mit den selben Werkzeugen erstellt werden. So kann im Notfall schnell gehandelt werden, um festzustellen, welche Systemkomponenten von einem Angriff gefährdet sind.
\newline Somit ist diese Arbeit im Bereich der IT-Security und Datensicherheit anzusiedeln. Dies sind in der heutigen Zeit wichtige Themenbereiche um den Schutz aller Daten bieten zu können.
\end{otherlanguage}

\lhead{Abstract}
\section*{Abstract}
\addcontentsline{toc}{section}{Abstract}



\newpage
\lhead{Abbildungsverzeichnis} 
\addcontentsline{toc}{section}{Abbildungsverzeichnis} 
\listoffigures

\newpage
\lhead{Tabellenverzeichnis}
\addcontentsline{toc}{section}{Tabellenverzeichnis} 
\listoftables
\newpage

\setlength{\parskip}{0.5em} 


%%%%%%%%%%%%%%%%%%%%%%%%%%%%%%%%%%
%%  Definition der Abkürzungen  %%
%%%%%%%%%%%%%%%%%%%%%%%%%%%%%%%%%%
\lhead{Abkürzungsverzeichnis} 
\section*{Abkürzungsverzeichnis} 
\addcontentsline{toc}{section}{Abkürzungsverzeichnis}  

\begin{acronym}
 \acro{VM}{Virtuelle Maschine}
\end{acronym}
\newpage
%%%%%%%%%%%%%%%%%%%%%%%%%%%%
%%  Glossar  %%
%%%%%%%%%%%%%%%%%%%%%%%%%%%%
\lhead{Glossar} 
%\section*{Glossar} 
\addcontentsline{toc}{section}{Glossar}  
\newpage
\begin{otherlanguage}{ngerman}
\section*{Glossar}
\begin{acronym}
 \acro{Admin(istrator)}{- ist ursprünglich der Verwalter eines Netzwerkes. Er hat die meisten Rechte und darf auf alle Einstellungen zugreifen}
 \acro{App}{- ist eine Anwendung}
 \acro{Backdoor}{- bezeichnet einen Teil einer Software, der es dem User ermöglicht, unter Umgehung der normalen Zugriffssicherung Zugang zum Computer oder einer sonst geschützten Funktion eines Programms zu bekommen}
 \acro{Browser Plugin}{- sind kleine Zusatzprogramme, die innerhalb von Browsern die Inhalte und Funktionen erweitern.}
 \acro{Cloud}{- ist in der IT die Abkürzung für Cloud Computing und wird auch als Rechnerwolke oder Datenwolke bezeichnet}
 \acro{Cloud Computing}{- beinhaltet Technologien und Geschäftsmodelle um IT-Ressourcen dynamisch zur Verfügung zu stellen und ihre Nutzung nach flexiblen Bezahlmodellen abzurechnen}
 \acro{Cloud Provider}{- Anbieter für Cloud Computing}
 \acro{Cloud Server}{- ist ein virtueller Server. Dieser bezieht seine Hardwarekomponenten von externen Dienstleistern und wird über das Internet angeboten.}
 \acro{Download}{ - das Herunterladen von Software}
 \acro{Dropper}{}
 \acro{Firewall}{- ist ein Sicherungssystem um Netzwerke und Computer vor unerwünschten Zugriffen zu schützen}
 \acro{Hardware}{}
 \acro{HashiCorp}{}
 \acro{Hashwert}{}
 \acro{Host-System}{}
 \acro{Infrastructure-as-Code-Tool}{}
 \acro{Key-Logger}{}
 \acro{Linker}{}
 \acro{Malware}{}
 \acro{Netzwerkressourcen}{}
 \acro{Open-Source-Software}{- ist Software deren Quellcode der Öffentlichkeit zugänglich ist. Die Benutzung des Codes steht dem Benutzer selbst offen.}
 \acro{Oracle}{}
 \acro{OS-Images}{- ist eine komprimierte Sammlung von Referenzdateien und Ordnern, die zum Installieren und Konfigurieren eines neuen Betriebssystems auf einem Computer verwendet werden}
 \acro{Provisioning}{}
 \acro{Rechenleistung}{}
 \acro{Remote-Access}{}
 \acro{Rootkits}{}
 \acro{Schadprogramm (Code)}{}
 \acro{Shell}{}
 \acro{Software}{}
 \acro{Speicherplatz}{}
 \acro{System Calls}{}
 \acro{Ubuntu}{}
 \acro{Wiki}{- ist eine Sammlung von Informationen und Beiträgen im Internet zu einem bestimmten Thema, die von den Nutzern selbst bearbeitet werden können}
 \acro{Workflow}{- die Abwicklung arbeitsteiliger Vorgänge}
\end{acronym}
\end{otherlanguage}
%%%%%%%%%%%%%%%%%%%%%%%%%%%%
%%  Einstellungen  %%
%%%%%%%%%%%%%%%%%%%%%%%%%%%%
\clearpage
\pagenumbering{arabic}  
    \setcounter{page}{1}
\lhead{\nouppercase{\leftmark}}

%%%%%%%%%%%%%%%%%%%%%%%%%%%%
%%  Hauptteil  %%
%%%%%%%%%%%%%%%%%%%%%%%%%%%%
\lhead{Einleitung}
\section{Einleitung} 
-
\subsection{Fragestellung}
-am ende entscheiden ob der punkt drin bleibt oder nicht
\subsection{Forschungsstand}
-nennung von Messen wie bspw. defcon
\newline
-publikationen aus exposé-quellen einbeziehen
\subsection{Aufbau der Arbeit}
-von welchem Thema gehe ich zu welchem thema um roten faden darzulegen
\newpage

\lhead{Theoretische Grundlagen}
\section{Theoretische Grundlagen} 
\subsection{Terraform}
\newpage
\begin{otherlanguage}{ngerman}
\subsection*{Funktionsweise}
\textit{Terraform} ist eine \textit{Open-Source-Software}, welche die Vorbereitung von \textit{Cloud Servern} einfacher macht. Sie wurde von HashiCorp dazu entwickelt, Infrastrukturen vorzubereiten und zu verwalten.\footcite{introform} Mit Infrastrukturen sind hier \textit{virtuelle Maschinen} verschiedenster Anbieter gemeint.\footnote{\cite{Terraform}} Terraform hat lizenzierte Partner für die seine Dienste anwendbar sind.\footcite{TerraProviders} Durch die hohe Konnektivität dieser Software sind die Einsatzorte sehr vielseitig. Das macht die Software natürlich auch in der Entwicklung und im Betrieb innerhalb von Unternehmen sehr attraktiv. 
\end{otherlanguage}

\subsubsection{Vorteile}
-schnell
\newline
-variabel
\newline
-nicht allzu komplex
\subsection{Virtuelle Maschinen}
\subsubsection{Was ist eine virtuelle Maschine?}
-Anwendungsumgebung
\newline
-Hardware imitiert

\subsubsection{Erstellen einer virtuellen Maschine}
-terraform einbringen und allgemeines vorgehen beschreiben wie es ohne terraform wäre
\newline
-überleitung absicherung
\subsubsection{Absicherung einer virtuellen Maschine}
-firewall 
\newline
-was für absicherungen sollten getroffen werden
\newline
-firewall konfiguration erklären
\subsection{Malware}
\subsubsection{Was ist Malware?}
-Schadcode 
\newline
-wurde dazu entwickelt schaden anzurichten
\newline
-nicht politisch werden!!! (nicht gru mit einbeziehen)
\subsubsection{Arten von Malware}
-virus
\newline
-Würmer
\newline
-trojaner
\newline
-Spyware
\newline
-Scareware
\newline
-Ransomware
\newline
-alle kurz anschneiden aber nicht zu tief
\subsection{Malware-Analyse}
- in den unteren 4 Punkten die Malwareanalyse erklären und auch den forschungsstand mit einbeziehen
\subsubsection{Einfache statische Analyse}
\subsubsection{Einfache dynamische Analyse}
\subsubsection{Erweiterte statische Analyse}
\subsubsection{Erweiterte dynamische Analyse}
\newpage

\lhead{Methodik}
\section{Methodik}
\subsection{Terraform}
- configs zeigen und einzelne befehle erklären: warum mache ich was
\subsection{Virtuelle Umgebung}
-was habe ich wo und wie gemacht
\newline
-mit screenshots dokumentieren und untermauern an passenden stellen
\subsection{Tools}
- UNBEDINGT RANSETZEN
\subsection{Malware}
- kurze Erklärung, dass ich Malware hab
\newpage

\lhead{Ergebnisse}
\section{Ergebnisse}
\newpage

\lhead{Bewertung der Ergebnisse}
\section{Bewertung des Ergebnisses}
\subsection{Vorteile der Vorgehensweise}
\subsection{Nachteile der Vorgehensweise}
\newpage

\lhead{Diskurs}
\section{Diskurs}
\subsection{Schlussbetrachtung}
\subsection{Ausblick}
\newpage 


\lhead{Literatur- und Quellenverzeichnis}
\section{Literatur- und Quellenverzeichnis}
%%%%%%%%%%%%%%%%%%%%%%%%%%%%
%% Literaturverzeichnis wird 
%% automatisch eingefügt
%%%%%%%%%%%%%%%%%%%%%%%%%%%%

\printbibliography
\addcontentsline{toc}{section}{Literatur- und Quellenverzeichnis}
aufteilen in internetquellen
publications
books

\newpage
\lhead{Anhang}
\section{Anhang}
\appendix
\section{Anhang A} 





%%%%%%%%%%%%%%%%%%%%%%%%%%%%
%% Eidesstattliche Erklärung
%%%%%%%%%%%%%%%%%%%%%%%%%%%%
\clearpage
\input{Erklaerung.tex}

\end{document}
