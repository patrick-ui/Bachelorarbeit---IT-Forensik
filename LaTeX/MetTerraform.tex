\begin{otherlanguage}{ngerman}
\section{Methodik}
\subsection{Virtuelle Umgebung}\label{Praktisches Terraform}
Um eine sichere virtuelle Umgebung zu schaffen, werden zwei Server benötigt. Ein Server bearbeitet DNS-Anfragen und der andere Server ist mit Malwareanalyse-Tools ausgestattet. Zudem sind keine Verbindungen zugelassen, außer Port 22. Dieser ist der SSH-Port und bietet hier die Möglichkeit den Server von außen zu beobachten. Ansonsten besteht die Möglichkeit für den Analyseserver sich mit dem Server zu verbinden der ein Netzwerk simuliert. Das ist daher nötig, da Malware teilweise nur mit Internetverbindung funktioniert. 
\subsubsection{Konfiguration der Umgebung}
Wie bereits beschrieben besteht die Cloud-Umgebung aus zwei Servern und einem Rechner, der Zugriff auf die Server hat.
\begin{figure}[h!]
    \centering
    \includegraphics[scale=0.7]{LaTeX/graphic/virtuelleUmgebung.png}
    \caption{Aufbau der Cloud-Umgebung}
\end{figure}
\newline
In Abbildung zwei ist dieser Aufbau bildlich dargestellt. Auf der rechten Seite befindet sich die virtuelle Umgebung (mit null gekennzeichnet). Auf ihr laufen die beiden virtuellen Server. Einer der Server ist für die Malware-Analyse zuständig (mit zwei gekennzeichnet). Hier sind die Tools vorzufinden und die Malware wird hier eingeschleust. Außerdem gibt es den zweiten virtuellen Server, welcher ein Netzwerk simuliert (mit drei gekennzeichnet). Dafür wird das Programm \dq iNetSim \dq{} verwendet. Dieses simuliert gängige Internetdienste und ist dafür gemacht bei der Malware-Analyse eingesetzt zu werden.
\newline
Diese beiden Server können über jeden Port miteinander kommunizieren, haben aber keine Verbindung nach außen, sondern bleiben in der geschlossenen Umgebung. Die Kommunikation der beiden Server ist in der Abbildung mit fünf gekennzeichnet. Die einzige weitere Verbindung die in diesem System hergestellt wird ist die Verbindung vom PC des Benutzers auf den Malware-Analyseserver über SSH (mit vier gekennzeichnet). Somit wird gewährleistet, dass man als Analyst der Malware das System steuern und beobachten kann. 
\newline
Anhand des Aufbaus der Umgebung ist auch zu erkennen wie abgekapselt das System ist. Es bestehen keine Verbindungen nach außen, sondern nur untereinander, bzw von außen nach innen. Das ist auch wichtig um das Ausdringen der Malware zu verhindern. Diese Regeln für die Kommunikation sind in den Firewallregeln festgehalten. 
\subsection{Terraform}
\subsubsection{Verbindung mit dem Cloud-Provider}
Um Terraform als Provisioning-Tool für Cloud-Umgebungen zu verwenden, muss eine Verbindung zwischen der Terraform-Umgebung und dem Provider vorhanden sein. Diese Verbindung wird über ein API-Token geschaffen. Ein API-Token ist
\subsubsection{Konfiguration der Server}
In Terraform können Server erstellt und entsprechend konfiguriert werden. Wie das gemacht wird, wird hier einmal aufgegriffen. 
\newline
In HCL werden Blocks für die Erstellung von vielen Komponenten verwendet, darunter auch für die Servererstellung. 
\tt  
\begin{lstlisting}[caption = Konfiguration Malware-Server, language=python, numbers=left, numberstyle=\tiny]
# Create a server
resource "hcloud_server" "malware" {
  name        = "malware"
  image       = "ubuntu-22.04"
  server_type = "cx11"
  firewall_ids = [hcloud_firewall.saferfw.id]
  ssh_keys = [hcloud_ssh_key.default.id]
  user_data = file("user_data.yml")

public_net {
    ipv4_enabled = true
    ipv4 = hcloud_primary_ip.primary_ip_Malware.id
    ipv6_enabled = false
  }
}
\end{lstlisting}
\rm
Der abgebildete Code ist die Konfiguration für einen Server in der \dq Hetzner-Cloud\dq. Das ist bereits in Zeile zwei hinter \dq resource \dq{} zu erkennen. Der hier erstellte Server trägt den Namen \dq malware \dq{} und hat als Betriebssystem \it Ubuntu \rm (Vgl. Zeile vier). Der Server-Typ (Zeile 5) legt fest, welche Cloud-Lösung des Providers verwendet werden soll. In der \dq Hetzner-Cloud \dq{} ist \dq cx11\dq{} die günstigste Version. Diese ist für die grundlegende Malware-Analyse dennoch ausreichend. Falls die Analyse durch stärkere Systemkomponenten beschleunigt werden soll, kann hier der Typ geändert werden. 
\newline
Im Anschluss daran wird dem Server eine Firewall zugewiesen. Welche Einstellungen diese hat wird im nächsten Unterpunkt genauer betrachtet. Daraufhin bekommt der Server die Info woher der SSH-Key genommen wird (Zeile sieben). In Zeile zehn werden die öffentlichen Netzwerkeinstellungen vorgenommen. Dazu wird festgelegt, welche \dq Internet-Protokoll-Version \dq{} vorhanden ist. In diesem Fall ist \dq IPv4 \dq{} erlaubt und \dq IPv6 \dq{} nicht. Die \dq IPv4 \dq-Adresse wird dem Server fest zugewiesen. Das ist für die spätere Kommunikation mit dem anderen Server wichtig.
\newline
\newline
\tt  
\begin{lstlisting}[caption = Konfiguration iNetSim-Server, language=python, numbers=left, numberstyle=\tiny]
#Create second Server
resource "hcloud_server" "iNetSim" {
  name        = "iNetSim"
  image       = "ubuntu-22.04"
  server_type = "cx11"
  firewall_ids = [hcloud_firewall.iNetSimFW.id]
  ssh_keys = [hcloud_ssh_key.default.id]
  
  public_net {
    ipv4_enabled = true
    ipv4 = hcloud_primary_ip.primary_ip_iNetSim.id
    ipv6_enabled = false
  }
}
\end{lstlisting}
\rm
Die Konfiguration des zweiten Servers ist, wie hier zu sehen, sehr ähnlich zu der des ersten Servers. Der Unterschied hier ist nur die zugewiesene Firewall (Zeile sechs) und die IP-Adresse (Zeile elf). 
\subsubsection{Konfiguration der Firewall}
Die Erstellung einer sicheren und durchdachten Firewall ist bei der Arbeit mit Malware sehr wichtig, um Schäden am eigenen oder fremden Systemen zu vermeiden. 
\newline
\newline
\tt
\begin{lstlisting}[caption = Konfiguration der Firewall für den Analyseserver, language=python, numbers=left, numberstyle=\tiny]
#Create a firwall
resource "hcloud_firewall" "saferfw" {
  name = "saferfw"
  rule {
    direction = "in"
    protocol  = "tcp"
    port      = "22"
    source_ips = [
      "0.0.0.0/0",
      "::/0"
    ]
  }

  rule {
    direction = "out"
    protocol  = "udp"
    port = "1-65535"
    source_ips = [
      hcloud_primaryip.primary_ip_iNetSim.id
    ]
  }

  rule {
    direction = "out"
    protocol = "tcp"
    port = "1-65535"
    source_ips = [
      hcloud_primaryip.primary_ip_iNetSim.id
    ]
  }
}
\end{lstlisting}
\rm
Dieser Code stellt die Definition der Firewall für den Malware-Analyseserver dar. Der Name dieser Firewall ist \dq saferfw \dq{} und sie besteht aus drei Regeln. Die erste Regel (Zeile vier bis elf) legt die Kommunikation über den SSH-Port fest. Diese ist nur nach innen zugelassen. Wer diese Verbindung zum Server herstellt ist egal, solange der SSH-Key stimmt. 
\subsection{Tools}
Für eine ausführliche Analyse der vorliegenden Malware werden Tools benötigt. Ziel ist hierbei, dass die Tools die statische und dynamische Analyse abdecken. So kann die Malware in der Umgebung vollumfänglich untersucht werden. Für die Analyse von Malware in Linux-Systemen gibt es bereits viele Möglichkeiten. Betriebssysteme wie \dq Kali-Linux\footnote{} \dq{} oder auch \dq Remnux\footnote{} \dq{} sind Linux-Varianten, die speziell für die Analyse von schädlichen Programmen ausgelegt sind. Hier sind bereits viele Tools vorinstalliert. Obwohl die Arbeit mit den eben genannten Betriebssystemen die Auswahl der Tools vereinfachen würde, werden diese hier nicht verwendet. Viele Cloud-Provider bieten für virtuelle Server diese Betriebssysteme nämlich nicht an. Stattdessen wird auf den hier erstellten virtuellen Servern Ubuntu als Linuxdistribution verwendet. Dieses ist bei vielen Cloud-Anbietern verfügbar und unterstützt somit das Ziel der Flexibilität der Umgebung. 
\newline
Als erstes geht es um die Tools für die statische Analyse. \dq TrID \dq{} ist zum Identifizieren von Dateien sehr nützlich. Die Daten, mit denen festgestellt wird, um was für eine Art von Datei es sich handelt werden aus einer Datenbank bezogen, die mit dem Programm verknüpft ist. Diese wächst immer weiter, sodass hier nicht mit veralteten Datenbanken gearbeitet wird. Das sichert die Verlässlichkeit der Daten. Zum Zeitpunkt der Recherche wird dieses Tool immer noch regelmäßig aktualisiert, was die Aktualität der Datenbanken unterstreicht.\footcite{TrID} 
\newline 
Für die erweiterte statische Analyse werden Disassembler und Decompiler verwendet. Diese helfen dabei, die Funktionsweise der schädlichen Datei zu verstehen. Dafür wird die Maschinensprache die der Computer interpretieren kann in eine für Menschen lesbare Sprache umgewandelt. Das Programm \dq Ghidra \dq{} ist für diesen Zweck sehr geeignet. Ghidra wurde ursprünglich für den Geheimdienst der USA entwickelt und ist nun als Open-Source-Software erhältlich. Es interpretiert die Maschinensprache und erzeugt aus einem ausführbaren Programm einen Code um, der für Menschen verständlich ist. So kann die Vorgehensweise des zu untersuchenden Programms analysiert und verstanden werden.\footcite{}
\newline
Als Debugger wird in dieser Cloud-Umgebung \dq GNU Debugger (GDB) \dq{} benutzt. Dies ist ein weiteres Tool, welches in einer Linuxumgebung für die Analyse von Schadcode verwendet werden kann. Es wird für die dynamische Analyse verwendet. \dq GDB \dq{} wird dafür verwendet, Programme zu starten und festzustellen, was das verhalten des gestarteten Programms beeinflussen kann. Außerdem können Bedingungen für den Start und Stop des Programmes festgelegt werden. Zudem wird dokumentiert was das Programm gemacht hat, sobald es gestoppt wurde.\footcite{}
\newline


\subsection{Wie die Malware auf die virtuelle Umgebung gelangt}
Die virtuelle Umgebung soll zur gezielten Analyse von Malware-Samples verwendet werden. Anders als bei \it Honeypods \rm, die Malware \dq anlocken \dq{}, gelangt die Schadsoftware also nicht zufällig auf das System, sondern muss gezielt heruntergeladen werden. 
\newline
Zuerst stellt sich die Frage, woher man die Malware bekommt. Wird sie einem zugeschickt mit der Aufgabe sie zu untersuchen ist sie meist bereits verpackt und über einen Download-Link abrufbar. Wird die Malware auf einem System gefunden und man möchte sie untersuchen, sollte man die zum Schadcode gehörenden Dateien verpacken. Dazu eignet sich das Dateiformat \dq Tar \dq{} sehr gut. Bei Dateien mit der Endung \dq .tar \dq{} werden alle zu verpackenden Dateien in einer Datei zusammengeführt. Der Inhalt dieser Datei ist erst nach dem entpacken wieder sichtbar. 
\newline
Wenn das Malware-Sample verpackt ist, bietet es sich an die Malware über einen Download-Link bereitzustellen. In dieser Arbeit ist diese Plattform \dq Seafile \dq. Auf der Cloud-Umgebung ist die Malware anschließend über das Kommando: \newline \tt wget [DOWNLOAD-LINK]\rm \newline herunterladbar. Dies kann nur erfolgen, wenn die Firewall ausgeschaltet ist, da in dieser die Kommunikation in das Internet verboten ist. Ist der verpackte Schadcode auf der Cloud-Umgebung, muss die Firewall wieder aktiviert werden. Anschließend kann die Malware für die Analyse entpackt werden. Dies geschieht unter Linux mit dem Kommando: \newline \tt tar -xvzf [NAMEDERDATEI]. tar. gz \rm \newline
Ab diesem Zeitpunkt ist die Malware auf der Analyseumgebung angekommen und ist bereit für die Analyse.
\newpage
\end{otherlanguage}