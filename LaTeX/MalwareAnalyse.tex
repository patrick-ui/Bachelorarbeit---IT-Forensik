\begin{otherlanguage}{ngerman}
\subsection{Malware-Analyse}
Malware kann sowohl statisch als auch dynamisch analysiert werden.
\subsubsection{Statische Analyse}
Bei der statischen Analyse wird das \it Sample \rm der Malware analysiert ohne es auszuführen. Oft ist das auch der erste Schritt bei der Analyse um eine Vorstellung darüber zu haben wie sich die Schadsoftware verhalten könnte. Es gibt hierfür verschiedene Tools und Vorgehensweisen. 
\newline 
Für diese Art der Analyse werden beispielsweise Antivirus-Programme verwendet. Diese können in der Regel bestätigen ob es sich bei dem vorliegenden Programm um Malware handelt. Um aussagekräftige Ergebnisse zu bekommen müssen hier mehrere Antivirus-Programme benutzt werden. Virusscanner können eine Aussage über die Malware treffen indem sie den \dq Fingerabdruck \dq{} der Datei mit den Fingerabdrücken in ihrer Datenbank vergleichen. Ein solcher \dq Fingerabdruck\dq{} ist einzigartig für jedes Programm. Er lässt sich in einem Hashwert abbilden. Um einen Hashwert zu ermitteln stehen verschiedene Hashfunktionen zur Verfügung. Darunter zählen beispielsweise der \dq Message-Digest-Algorithm 5\dq (MD5) oder der \dq Secure Hash Algorithm\dq{} (SHA) in verschiedenen Ausführungen. Das Problem an Hashfunktionen ist, dass je nach Inhalt der zu verschlüsselnden Datei ein neuer Wert ermittelt wird. Das heißt, wenn der Programmierer des Schadcodes einen Part des Codes abändert, ist der Hashwert ein anderer. So hat sich an der Malware nichts verändert außer der Hashwert. Dadurch kann es vorkommen, dass Virenscanner Malware aufgrund minimaler Änderungen nicht mehr erkennen können. Somit erkennt die Antivirusprüfung Malware nicht mit völliger Sicherheit. Um Antivirus-Programme zu täuschen verwenden Hacker beispielsweise \it msfvenom\rm. Dieses Tool hilft bei der Verschlüsselung von Code. So kann die selbe Malware einen anderen Hashwert haben. Unter Umständen ist dieser auch nicht in den Datenbanken der Virenscanner enthalten, wodurch die schädliche Software unentdeckt bleiben kann.
\newline
Um in der Analyse weiter voranzukommen und herauszufinden wie sich die Malware verhält gibt die Methode Strings aus dem Code zu lesen und zu deuten. Strings sind in der \it IT \rm Zeichenketten. Diese sind in bestimmten Fällen im Code von ausführbaren Dateien, beispielsweise als \it URLs \rm, enthalten. Sie können Auskunft über die Funktionsweise der Malware geben und sind somit ein wichtiger Faktor in der Analyse dieser. Um Strings aus Dateien auslesen zu können müssen die entsprechenden Tools installiert sein. Anhand der Menge der gefundenen Strings kann vermutet werden ob ein Programm verschachtelt oder verpackt wurde. Das ist eine Methode die Hacker verwenden um in einer Antivirus-Prüfung nicht aufzufallen. Womit das Programm verpackt wurde kann unter Linux anhand der Software \dq Detect-It-Easy\dq{] aufgedeckt werden. Solche Programme geben die Art des sogenannten \dq Packers\dq{} aus. So kann die Malware wieder entpackt werden um mehr Strings finden und auszuwerten zu können.
\newline 
Zur grundlegenden statischen Analyse von Malware gehört auch zu überprüfen ob verlinkte Bibliotheken oder Funktionen vorliegen. Diese Verbindungen zu verschiedenen Bibliotheken können sowohl statisch als auch dynamisch sein. Statisch bedeutet hierbei, dass der gesamte Code der Bibliothek in den Quelltext kopiert wird. Diese Methode wird jedoch selten verwendet. Öfter kommt die dynamische Verlinkung zu Bibliotheken zum Einsatz. Der Begriff \dq dynamisch \dq{} kommt daher, dass die Verbindung erst im laufenden Prozess hergestellt wird. Diese Verlinkungen können weitere Aussagen darüber treffen, wie sich die Malware im laufenden Zustand verhält.
\subsubsection{Dynamische Analyse}
Die dynamische Analyse von Malware ist die Untersuchung des Schadprogrammes während es ausgeführt wird. Hierbei wird das Verhalten der Malware während der \it Laufzeit \rm beobachtet und analysiert. Außerdem ist die Analyse des Systems nach der Laufzeit ein Teil dieser Methode. Das ist nützlich um den Schadcode einer groben Art von Malware zuzuordnen. Wichtig hierbei ist vorallem die Umgebung auf der das Schadprogramm ausgeführt wird. Es ist wichtig eine Art \it forensiches Labor \rm zu erstellen in der kein Schaden für das Produktionssystem entstehen kann. 
\newline
Um eine Untersuchung durchzuführen muss das Programm erstmal ausgeführt werden. Wenn die ausführbare Datei nicht im Format \dq .exe\dq{} vorliegt werden hierfür, je nach Betriebssystem, weitere Tools benötigt. Wenn 


\end{otherlanguage}